% !TEX TS-program = xelatex
% !TEX encoding = UTF-8 Unicode
%%The above fixes a unicode issue which was preventing Japanese text from being rendered. It has to be there commented out. Don't be fooled into deleting it just because it appears to be commented out. That's just how it works. Copy-pasted from internet forums. 

\documentclass{article}
\usepackage{fullpage}
\usepackage{nopageno}
\usepackage{tikz}
\usetikzlibrary{calc, decorations.pathmorphing, decorations.markings, patterns}

%Japanese text 
\usepackage{xeCJK}
\usepackage{ruby}
\setCJKmainfont{Hiragino Mincho Pro}
\renewcommand{\rubysep}{0.1pt}

%\usepackage{xeCJK}
%\setCJKmainfont{AozoraMinchoRegular.ttf}
%\setCJKsansfont{KodomoRounded-Light.otf}
%\setCJKmonofont{KodomoRounded-Light.otf}


\title{Notes on KamLAND mini-balloon construction}
\date{June-July 2015}
\author{Emmett Krupczak}
% Version 2.3 - July 22, 2015
%Pulled from ek-edits and now edited by ek. Hurray! 

%Next: Add pictures! 

\begin{document}
\maketitle

%%Undone sections: Materials, clean room, checking 

\section{Materials}
\begin{itemize}
\item Blue pen, tested to ensure low radioactivity. (Brand??)
\item Flashlight (purple; maybe UV?)
\item Plastic bins to store film, with nitrogen gas tubing
\item Cutting boards, scalpels, rulers, meter sticks
\item Rollers for flattening creases out of film
\item Weights to hold down patterns
\item Tape to label storage bins
\end{itemize}
Equipment included: ultrasonic cleaner, welder, nitrogen gun, static electricity eliminator bar, floor-mounted static electricity eliminators, particle detector. 

\section{Clean room procedures}
When washing and preparing items in the less-clean clean room, we simply wore clean suits over our day clothes (suit, hood, hair net, mask, shoes, two pairs of gloves.) 

When working with the thin film directly, we changed out of our day clothes into special ``clean wear.'' (Non-linting shirt and pants. The shirt and pants were white and slightly transparent.) Over this, we put on our normal clean suit attire. After entering the clean room, we entered a small section enclosed with hanging nylon curtains and changed into a new clean suit which had a hood that covered our whole face. This went on over our clean wear, hair net, mask, and first pair of gloves. We then put on new clean shoes, a new outer pair of gloves, and goggles. This left no skin exposed. 
The special enclosed changing area was carefully kept isolated, with separate chambers for changing out of the normal clean suits vs changing into the new clean suits. The floor was wiped at the end of the day. 

A section was cut out of some of the goggles to keep them from fogging up. This seemed to be particularly a problem for people who wore glasses. The clean suit packages were washed before they were brought into the inner changing room, and the goggles were washed after every use. Unlike the normal / outer clean suits, which were re-worn for several visits, the inner clean suits were washed after every use. The clean wear was washed at the end of every day (someone working a morning and an afternoon shift would wear the same clean wear for both shifts; otherwise a new package was opened each time.) 

Everything brought into the clean room was washed in ultra-pure water and sometimes also detergent.  Even screws were washed individually with a toothbrush. 
%Furniture was supposed to be washed with at least ``ten strokes.'' (or something like that??) 
Of particular trouble was furniture with unreachable hollow areas, and furniture with labels.  Useful equipment: Lots of sponges and toothbrushes. Sometimes the sponges and toothbrushes shed pieces, which was a drawback. 


\section{Cutting the films}
The films were cut into 7 m lengths. We originally planned to cut 90 films. We ended up cutting a total of about 85 films. (I'm not sure why the discrepancy.) 
%
In order to cut the films, we unrolled the film while holding it above a table. Two people were needed to unroll the large nylon roll in a controlled way. We went very slowly. Above the unrolling film was a static electricity eliminator bar (\ruby{除電}{じょでん}バー)The first film was checked for defects after being unrolled, and some small scratches were found, but the remainder of the films were not checked until after the washing stage.  Large pieces of dirt or films with serious defects, however, were noted.  We found a long stretch of film which had very regular black dots along the entire center. The dots appeared to be some kind of ink or oil. The dots gradually got lighter over several meters (I think at least 9 meters were so marked.) This section was removed, although one or two spotted films did slip through and were removed after the washing stage during the more thorough checking process. 
The films were then carefully folded accordion style and stacked for washing.

It took two people to unroll the film, three people per side (total of six) to hold the film as it was being unrolled, and two people to fold the film from the end. The folding began as soon as the film was unrolled and measured to be 7 m but before the film was cut, so cutting and folding happened simultaneously to save time. The folding was made much easier if the people holding the film kept the film taut. 

\section{Washing the films}
The films were washed in ultra pure water using a sonic cleaner. A pipe was carefully bent into a U-shape with a teflon roller at the bottom of the U. This held the film under the water as it was fed through the tank past the sonic cleaner. 
%What kind of sonic cleaner? Was there any optimization done here? 

It took three people to unfold the films, one using the 除電バーto remove static electricity as the film was being unfolded. This was sometimes the slowest part of the process. Then, two people would feed the film into the tank, two people would shepherd the film through the tank, and five people would hold the film as it left the tank, for a total of twelve people required. The film was removed very slowly from the tank, slowly enough that water droplets did not stick to the film as it was removed. 

After the film had finished its trip through the water tank, it was carefully hung from clothespins. A nitrogen gun was used to blow off any water drops that remained on the films. There were usually some water drops along the edges. This procedure took six people, two teams of three. Each team would start on opposite ends of the film, one team on each side. They would slowly move along the film past each other, blowing off drops that they found on their side of the film as they went. One person wielded the gun and two others would hold the film steady, from the bottom and from the top, respectively. 

Some pitfalls: The person holding the nitrogen gun could not touch the film because the nitrogen gun made their hands dirty. The cords and nitrogen lines from the guns tended to get tangled and only the person holding the gun or someone else with ``dirty'' hands could touch them to untangle them.  The nitrogen guns had to be turned on and off simultaneously because the two guns shared a nitrogen supply, so if only one gun was on, it had a much higher flow, and a high flow of nitrogen produced more dust out of the nitrogen guns.  

The amount of dust produced by the nitrogen guns and by the nitrogen-powered じょでんバー\mbox{ } was checked at the beginning of each shift. Sometimes it was very high - on one occasion we had to change the filter before using it. The dust may have been produced by the control valve (??). 

While the drying was happening, a large pole-mounted static electricity eliminator was slowly moved past the film.  Checking the film would also begin at this stage, and usually took longer than the nitrogen drying. 

\section{Checking the films}
Checking the films was done by eye, with the occasional assistance of a flashlight. %(A purple flashlight I think?)
Defects were marked using a blue pen. The purple flashlight was used to highlight potentially damaged areas; specks of dirt and creases/scratches could be made to shine brightly if the light was put on them at the right angle.  (Note: when the light was shone on the film, I would often see bright spots which looked like specks of dirt, but which were not visible without the light. We did not mark these specks. Even if we had wanted to, they were much too prevalent.) 

 Originally, the major defect noticed were large creases in the film that appeared to potentially be scratches or cuts. Upon microscope examination, however, they turned out to be creases rather than cuts, and pull-testing seemed to indicate that the creased films did not have an excessively weakened breaking strength. 
These large creases were present on nearly all films and appeared in a symmetrical, repeated way, diagonally along the top and bottom of the film. They were between 10 and 35 cm long (my estimate; we did not measure).  Because of their prevalence, we stopped marking them once it was determined that they were not actual cuts in the film and did not weaken the film.

The other major two types of things we identified were dirt \emph{on} the film, and dirt \emph{in} the film.  Dirt on the film was rare, maybe one every three-five films, and it tended to be larger - up to 1-2 mm in diameter.  We were able to remove some dirt from the film but we usually marked those areas as contaminated even if the dirt was removed.  Dirt in the film was the main form of contamination. The dirt varied in size and color; a typical example would be black and about 0.5 mm in diameter, but some specks were white and some were as small as a pin prick. They were very hard to see. The number varied but there would typically be about 1 per meter, with some 7 m films having as many as 15 specks.  Because they were so prevalent, we began to run out of films that did not have any dust specks marked in the center. So, the criteria for marking a speck tightened as we neared the end of the films, requiring the specks to be bigger. There was not a consistent cutoff point; it varied by film. If a film had already been determined to be unusable, we would sometimes return and mark dust specks we had previously allowed to pass. This was the most difficult part of the marking process. 

Note: It might have been helpful to have a way to mark dirt specks that didn't make the film unusable - once the marker was on the film, we could not use that part of the film even if we changed our mind about the dirt. This would have allowed us to mark freely and then pick the most clean films at the end. 

Besides the dirt and the long creases, the films sometimes had scratches or pinches. These were usually very short - about 1 cm long or less - and sometimes occurred in small clusters. We usually marked them if they seemed severe, but it was very much a case-by-case decision.  Some films also had opaque ``spots'' or patches. These were usually marked. On one particularly strongly affected film, we marked only the largest spots. One film had a smudge (maybe grease?) and one film had red end-of-roll marks. 

After the number of dirt pieces or other defects (キズ)  on the film was marked, the number of dirt pieces (ゴミ) was written on the edge of the film. A diagram of the film was drawn on a piece of tape showing the approximate relative locations of the dirt pieces.  This piece of tape was stuck on the top of the bin that the film was to be stored in. The number of dirt pieces and the usability of the film (i.e. any dirt pieces too close to the center to cut out a complete gore pattern ) was written on the outside of the box, along with its location in the box (i.e. 2nd from the bottom, etc.) 

The film was then carefully folded. The marker had a tendency to come off and stick to other parts of the film. This problem wasn't totally realized at the time and would cause issues later when previously clean film became contaminated with marker. But, we at least folded down an extra flap of film over the label marking the number of dirt pieces to prevent it from stamping marker on clean film. The film was then folded up accordion style into a strip about 25 cm wide by three people with clean and dry hands. The folding was easiest if the film was held taut, and sometimes was assisted by a fourth person who could remove the clothes pins. It was important to fold the film carefully and slowly because having to undo and refold a section created static electricity. 

\section{Storing the films}
The films were stored in plastic bins that had nitrogen lines attached. The positive nitrogen pressure in the bins was to prevent dust  (or static electricity?) from building up on the films during storage. %%Why the nitrogen? 
The films were stored folded accordion style, and were stacked several to a bin. The number of dirt pieces and the usability of each film in the box was labeled on the outside of the box. Eventually, we started sorting the films by whether or not they were clean enough to be inner films, accumulating washed and checked films in two boxes simultaneously.  This seemed to be a better method of sorting and made it easier to do the next step: stacking. 

\section{Stacking the films}
The films were stacked into a ``sandwich'' of three films - a ``clean'' film in the middle and a ``dirty'' film on either side. (Clean films were those without any marked dirt spots that would prevent the cutting of an entire gore shape.) The dirty films were lain with their marked sides facing out. 
The tally of the number of dirt spots on each film was used to identify all the dirt spots on the outer (dirty) films. The markings around the dirt spots were then removed with ethanol. 
%%Digram of order of stacked films

Note: occasionally the tally was inaccurate and there would be an extra dirt spot - or a missing one. This made the checking take a little longer because we had to examine the entire film carefully even after we had found the expected number of markings. Sometimes we would find the missing dirt spots at the next stage of the process.

The three film sandwiches were held together very securely by the static electricity - it was hard to separate the films once they had been lain down. It was important that the middle film be as flat as possible (bubbles and creases made accurate marking and cutting harder), and to do this, one person carefully flattened each outer film onto the middle film. This was a very fiddly part of the process. Any redos created static electricity. To pull up and redo a large section, the じょでんバー was used to help separate the films. 

The three film sandwiches were marked with the piece of tape showing the number and locations of the dirt particles on the inner film, and were then carefully stacked on the third table. 


\section{Drawing the patterns} 
The three film sandwiches were transferred back to the main table and the pre-cut patterns were placed over the film. First, the patterns were outlined in marker and then the experts\footnote{The professors and post-docs were referred to as ``experts''} drew in guide lines. (On certain spots on the patterns, additional marks were made to help with lining up the pieces later - marking the center, etc.) Then, the pattern was removed and small rulers were used to make a dashed line 1.5 cm inside the edge of the pattern. This was to help guide the film overlap for welding. Some additional alignment marks were made by the experts. 

\section{Cutting out the patterns}
To cut the patterns, cutting boards were placed between the film and the table and a scalpel was used to carefully cut along the films. Because a ragged edge creates a place where the film can tear easily, the cuts were made in smooth strokes, beginning and ending at the edge of the film, so that there were no start/stop points on the final pattern. %%[Diagram of curved cutting technique]

Some of the patterns involved cutting out small quarter-sized circles. This was extremely difficult to do smoothly and had to be redone several times. The final result was still not quite satisfactory. (Note: If many of these had to be done, some kind of circular punch might be worth finding.) 

Besides the shapes of the gores and the end caps (called ``cone'' and ``polar''), we also had to cut twenty-four 1.5 cm wide strips, each 7 m long. These will be used in the welding process to lay over the seams. The seams will be a sandwich of two such strips and the two pieces of nylon being bonded (a layer of four.) 
In order to make these, we removed an outer sheet from two sets of three-film sandwiches in order to make a four-film sandwich (dirty, clean, clean, dirty) and removed the markings from the dirty films as before.  (There was one set where we appeared not to remove the markings from one of the dirty films - I am unclear on why that set was different, but the ultimate goal seemed to be the same.) 

To draw the long strips, we used meter sticks and rulers. A special cutting roller designed to simultaneously cut four 1.5 cm strips was tested, but it did not cut reliably enough, so the strips were drawn and (presumably?) cut by hand.  Drawing the strips was a fairly time-consuming process that would have been made easier with more longer meter sticks. 

\section{Welding the films} 
\subsection{Preparing the welding machine}
The welding machine was thoroughly wiped before being brought into the clean room. However, it is not ``rated'' for clean rooms and still sheds dust, especially when the welding arm is triggered. (The machine works by pressing a foot pedal which causes the welding arm to lower for a set period of time and with a set temperature and pressure.) To prevent this, the machine was wrapped in nylon at key points, mostly around the moving welding head.  This was done using the same thicker nylon that was used for the changing room and balloon case (see ``Other''). 

Then, extensive testing was done using the particle detector to determine if the welding machine was still shedding dust.  By adding more nylon and rechecking, we were able to get the welding machine into a state in which it released at most only a few particles per press. We then tested the temperature and time needed to get a clean weld. Bubbles often appeared in the weld if the pressure was not high enough. In the end, about 290$^\circ$ C (??) temperature and 2.0 second weld time was used. This still resulted in some bubbles but the machine was having problems allowing us to select a weld time longer than 2 s.  The welding machine also had a problem where sometimes the emergency stop sensor would be triggered and the welding arm would rapidly withdraw. This did not seem to be caused by the nylon wrapping around the machine.  (It might have happened more for higher pressure settings??) 

%Picture of welding machine

\section{Other}
We washed a large section of thicker nylon and a long nylon-mounted zipper. We then welded these together to create a case for holding the balloon. It also served as useful welding practice. We used a smaller hand-held welder for this, rather than the large welder that is intended for the thin film welding. 

\section{Lessons Learned (for future reference)}
\begin{itemize}
\item The blue pen stamps on the clean film sometimes, probably if you fold the film too soon after marking. This wastes otherwise clean film.
\item There are a lot of dirt particles inherent to the film from the manufacturing procedure. If every single one is marked, there probably won't be enough unmarked film left. But they should be marked. Conflict. 
\item The nitrile(?) gloves are probably shedding little dust particles onto the film. 
\item Waiting for gloved hands to dry after washing is slow and is sometimes actually the limiting factor on moving on to the next step. 
\item Using nitrogen to blow droplets off the film is very slow. The droplets don't move easily.
\item Have plenty of long meter sticks and cutting boards and cutting implements so marking and cutting can be as parallelized as possible. 
\end{itemize}



\end{document}
