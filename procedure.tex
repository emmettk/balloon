% !TEX TS-program = xelatex
% !TEX encoding = UTF-8 Unicode
%%The above fixes a unicode issue which was preventing Japanese text from being rendered. It has to be there commented out. Don't be fooled into deleting it just because it appears to be commented out. That's just how it works. Copy-pasted from internet forums. 

\documentclass{article}
\usepackage{fullpage}
\usepackage{nopageno}
\usepackage{tikz}
\usetikzlibrary{calc, decorations.pathmorphing, decorations.markings, patterns}

%Japanese text 
\usepackage{xeCJK}
\usepackage{ruby}
\setCJKmainfont{Hiragino Mincho Pro}
\renewcommand{\rubysep}{0.1pt}

%\usepackage{xeCJK}
%\setCJKmainfont{AozoraMinchoRegular.ttf}
%\setCJKsansfont{KodomoRounded-Light.otf}
%\setCJKmonofont{KodomoRounded-Light.otf}


\title{Notes on KamLAND mini-balloon construction}
\date{June-July 2015}
\author{Emmett Krupczak}
% Version 2.1 - July 22, 2015
%Pulled from ek-edits and now edited by ek. Hurray! 

\begin{document}
\maketitle

%%Undone sections: Materials, clean room, checking 

\section{Materials}
\begin{itemize}
\item Blue pen, tested to ensure low radioactivity. %Brand??
\item Flashlight %(UV / purple??)
\end{itemize}

\section{Clean room procedures}

\section{Cutting the films}
The films were cut into 7 m lengths. We originally planned to cut 90 films. We ended up cutting a total of about 85 films. (I'm not sure why the discrepancy.) 
%
In order to cut the films, we unrolled the film while holding it above a table. Two people were needed to unroll the large nylon roll in a controlled way. We went very slowly. Above the unrolling film was a static electricity eliminator (\ruby{除電}{じょでん}バー)The first film was checked for defects after being unrolled, and some small scratches were found, but the remainder of the films were not checked until after the washing stage.  Large pieces of dirt or films with serious defects, however, were noted.  We found a long stretch of film which had very regular black dots along the entire center. The dots appeared to be some kind of ink or oil. The dots gradually got lighter over several meters (I think at least 9 meters were so marked.) This section was removed, although one or two spotted films did slip through and were removed after the washing stage during the more thorough checking process. 
The films were then carefully folded accordion style and stacked for washing.

It took two people to unroll the film, three people per side (total of six) to hold the film as it was being unrolled, and two people to fold the film from the end. The folding began as soon as the film was unrolled and measured to be 7 m but before the film was cut, so cutting and folding happened simultaneously to save time. The folding was made much easier if the people holding the film kept the film taught. 

\section{Washing the films}
The films were washed in ultra pure water using a sonic cleaner. A pipe was carefully bent into a U-shape with a teflon roller at the bottom of the U. This held the film under the water as it was fed through the tank past the sonic cleaner. 
%What kind of sonic cleaner? Was there any optimization done here? 

It took three people to unfold the films, one using the 除電バーto remove static electricity as the film was being unfolded. This was sometimes the slowest part of the process. Then, two people would feed the film into the tank, two people would shepherd the film through the tank, and five people would hold the film as it left the tank, for a total of twelve people required. The film was removed very slowly from the tank, slowly enough that water droplets did not stick to the film as it was removed. 

After the film had finished its trip through the water tank, it was carefully hung from clothespins. A nitrogen gun was used to blow off any water drops that remained on the films. There were usually some water drops along the edges. This procedure took six people, two teams of three. Each team would start on opposite ends of the film, one team on each side. They would slowly move along the film past each other, blowing off drops that they found on their side of the film as they went. One person wielded the gun and two others would hold the film steady, from the bottom and from the top, respectively. 

Some pitfalls: The person holding the nitrogen gun could not touch the film because the nitrogen gun made their hands dirty. The cords and nitrogen lines from the guns tended to get tangled and only the person holding the gun or someone else with ``dirty'' hands could touch them to untangle them.  The nitrogen guns had to be turned on and off simultaneously because the two guns shared a nitrogen supply, so if only one gun was on, it had a much higher flow, and a high flow of nitrogen produced more dust out of the nitrogen guns.  

The amount of dust produced by the nitrogen guns and by the nitrogen-powered じょでんバー\mbox{ } was checked at the beginning of each shift. Sometimes it was very high - on one occasion we had to change the filter before using it. The dust may have been produced by the control valve (??). 

While the drying was happening, a large pole-mounted static electricity eliminator was slowly moved past the film.  Checking the film would also begin at this stage, and usually took longer than the nitrogen drying. 

\section{Checking the films}
Checking the films was done by eye, with the occasional assistance of a flashlight. %(A purple flashlight I think?)
Defects were marked using a blue pen.

 Originally, the major defect noticed were large creases in the film that appeared to potentially be scratches or cuts. Upon microscope examination, however, they turned out to be creases rather than cuts, and pull-testing seemed to indicate that the creased films did not have an excessively weakened breaking strength. 
These large creases were present on nearly all films and appeared in a symmetrical, repeated way, diagonally along the top and bottom of the film. They were between 10 and 35 cm long (my estimate; we did not measure).  Because of their prevalence, we stopped marking them once it was determined that they were not actual cuts in the film and did not weaken the film.

The other major two types of things we identified were dirt \emph{on} the film, and dirt \emph{in} the film.  Dirt on the film was rarer and tended to be larger - up to 1-2 mm in diameter.  We were able to remove some dirt from the film but we usually marked those areas. 



\section{Storing the films}
The films were stored in plastic bins that had nitrogen lines attached. The positive nitrogen pressure in the bins was to prevent dust  (or static electricity?) from building up on the films during storage. %%Why the nitrogen? 
The films were stored folded accordion style, and were stacked several to a bin. The number of dirt pieces and the usability of each film in the box was labeled on the outside of the box. Eventually, we started sorting the films by whether or not they were clean enough to be inner films, accumulating washed and checked films in two boxes simultaneously.  This seemed to be a better method of sorting and made it easier to do the next step: stacking. 

\section{Stacking the films}
The films were stacked into a ``sandwich'' of three films - a ``clean'' film in the middle and a ``dirty'' film on either side. (Clean films were those without any marked dirt spots that would prevent the cutting of an entire gore shape.) The dirty films were lain with their marked sides facing out. 
The tally of the number of dirt spots on each film was used to identify all the dirt spots on the outer (dirty) films. The markings around the dirt spots were then removed with ethanol. 
%%Digram of order of stacked films

Note: occasionally the tally was inaccurate and there would be an extra dirt spot - or a missing one. This made the checking take a little longer because we had to examine the entire film carefully even after we had found the expected number of markings. Sometimes we would find the missing dirt spots at the next stage of the process.

The three film sandwiches were held together very securely by the static electricity - it was hard to separate the films once they had been lain down. It was important that the middle film be as flat as possible (bubbles and creases made accurate marking and cutting harder), and to do this, one person carefully flattened each outer film onto the middle film. This was a very fiddly part of the process. Any redos created static electricity. To pull up and redo a large section, the じょでんバー was used to help separate the films. 

The three film sandwiches were marked with the piece of tape showing the number and locations of the dirt particles on the inner film, and were then carefully stacked on the third table. 


\section{Drawing the patterns} 
The three film sandwiches were transferred back to the main table and the pre-cut patterns were placed over the film. First, the patterns were outlined in marker and then the experts\footnote{The professors and post-docs were referred to as ``experts''} drew in guide lines. (On certain spots on the patterns, additional marks were made to help with lining up the pieces later - marking the center, etc.) Then, the pattern was removed and small rulers were used to make a dashed line 1.5 cm inside the edge of the pattern. This was to help guide the film overlap for welding. Some additional alignment marks were made by the experts. 

\section{Cutting out the patterns}
To cut the patterns, cutting boards were placed between the film and the table and a scalpel was used to carefully cut along the films. Because a ragged edge creates a place where the film can tear easily, the cuts were made in smooth strokes, beginning and ending at the edge of the film, so that there were no start/stop points on the final pattern. %%[Diagram of curved cutting technique]

Some of the patterns involved cutting out small quarter-sized circles. This was extremely difficult to do smoothly and had to be redone several times. The final result was still not quite satisfactory. \footnote{If many of these had to be done, some kind of circular punch might be worth finding.} 

%\section{Welding the films} 
%TBD!


\end{document}
